\documentclass{article}
\usepackage[utf8]{inputenc}
\usepackage{amsmath}

\title{Medium Test Document}
\author{Test}
\date{\today}

\begin{document}

\maketitle
\tableofcontents

\section{Introduction}
This is a medium-sized test document that exercises several LaTeX features.
We include math, lists, tables, and cross-references to test compilation performance.

\section{Mathematics}

\subsection{Inline Math}
The Euler identity $e^{i\pi} + 1 = 0$ is considered one of the most beautiful equations.

\subsection{Display Math}
\begin{equation}
  \int_{-\infty}^{\infty} e^{-x^2} dx = \sqrt{\pi}
  \label{eq:gaussian}
\end{equation}

As shown in Equation~\ref{eq:gaussian}, the Gaussian integral has a closed form.

\begin{align}
  \nabla \cdot \mathbf{E} &= \frac{\rho}{\varepsilon_0} \\
  \nabla \cdot \mathbf{B} &= 0 \\
  \nabla \times \mathbf{E} &= -\frac{\partial \mathbf{B}}{\partial t} \\
  \nabla \times \mathbf{B} &= \mu_0 \mathbf{J} + \mu_0\varepsilon_0 \frac{\partial \mathbf{E}}{\partial t}
\end{align}

\section{Lists}

\begin{enumerate}
  \item First item with explanation
  \item Second item with subitems:
    \begin{itemize}
      \item Sub-item A
      \item Sub-item B
      \item Sub-item C
    \end{itemize}
  \item Third item
\end{enumerate}

\section{Tables}

\begin{table}[h]
\centering
\begin{tabular}{|l|c|r|}
  \hline
  Left & Center & Right \\
  \hline
  One & Two & Three \\
  Four & Five & Six \\
  Seven & Eight & Nine \\
  \hline
\end{tabular}
\caption{A sample table}
\label{tab:sample}
\end{table}

See Table~\ref{tab:sample} for details.

\section{Conclusion}
This document tests cross-references, math environments, lists, and tables.

\end{document}
